\documentclass[11pt]{article}
\renewcommand{\baselinestretch}{1.0}

\usepackage{amsfonts}
\usepackage{hyperref}
\usepackage{color}
\usepackage{graphicx}
\usepackage{fullpage}
\usepackage{booktabs}
\usepackage{amsmath}
\usepackage{amssymb}
\pagestyle{empty}
\renewcommand{\baselinestretch}{1.0}

\title{Statistics Project 1 : \\ Statistics - Math 2606}
\author{Kevin Chen, William Gantt, Nick Barnes}
\date{}
\begin{document}
\maketitle
\section{Experiment}
\section{Results}
\begin{figure}
\centering
\begin{tabular}{ccccc}
\toprule
Measurement \# & $h_1$ & $d_1$ & $d_2$ & $h_2$ \\
\midrule
1 & 1.78 & 1.12 & 10.86 & 17.26 \\
2 & 1.77 & 1.2 & 11.04 & 16.28 \\
3 & 1.76 & 1.25 & 11.57 & 16.29 \\
4 & 1.77 & 1.41 & 12.41 & 15.58 \\
5 & 1.77 & 1.46 & 13.49 & 16.35 \\
6 & 1.77 & 1.66 & 14.76 & 15.74 \\
7 & 1.76 & 1.69 & 16.18 & 16.85 \\
8 & 1.77 & 1.84 & 17.72 & 17.05 \\
9 & 1.78 & 1.87 & 19.34 & 18.41 \\
10 & 1.77 & 2.25 & 21.02 & 16.54 \\
\midrule
mean & 1.77 & 1.575 & 14.839 & 16.635 \\
\bottomrule
\end{tabular}
\caption{Measurements for iris height ($h_1$), the distance from the mirror to Will ($d_1$), the distance from Searles to the mirror ($d_2$), and the resultant estimate of the height of Searles ($h_2$). All units are in meters.}
\end{figure}
Our estimates for the standard deviation of the measurements, up to an order of magnitude, are as follows:
\begin{align*}
\sigma_{h_1} &= 1\text{cm} \\
\sigma_{d_1} &= 1\text{dm} \\
\sigma_{d_2} &= 1\text{m}
\end{align*}


\end{document}